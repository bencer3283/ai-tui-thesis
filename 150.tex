\documentclass[12pt]{article}
\usepackage[letterpaper, total={16cm, 22cm}]{geometry}
\usepackage[stable]{footmisc}
\usepackage{titlesec}
\usepackage{graphicx}
\usepackage{float}
\usepackage{multicol}
\usepackage{enumitem}
\usepackage{array}
\usepackage{longtable}
\usepackage{vcell}
\setlength{\columnsep}{1cm}
\PassOptionsToPackage{hyphens}{url}\usepackage{hyperref}
% \usepackage{fancyhdr}
% \setlength{\headheight}{15pt}
%\pagestyle{fancy}
% \fancyhead[L]{Hidden Cost of Technology}
% \fancyhead[R]{Ben Cheng}
\usepackage{amsmath}
\usepackage{latexsym}
\usepackage{multirow}
\graphicspath{ {./images/} }
\usepackage[backend=biber, style=authoryear, sorting=none]{biblatex}
\addbibresource{ref.bib}
\usepackage{mathspec}   %加這個就可以設定字體
%\usepackage{xeCJK}       %讓中英文字體分開設置
%\setCJKmainfont{Noto Serif CJK TC} %設定中文為系統上的字型
%\newCJKfontfamily[chineseSans]\CJKsans{Noto Sans CJK TC}
\setmainfont{IBM Plex Serif}
\setsansfont{Space Grotesk}
\setmonofont{IBM Plex Mono}
%\renewcommand{\familydefault}{\sfdefault}
\XeTeXlinebreaklocale "zh"             %這兩行一定要加,中文才能自動換行
\XeTeXlinebreakskip = 0pt plus 1pt     %這兩行一定要加,中文才能自動換行
\renewcommand{\baselinestretch}{1.25}
% \renewcommand{\figurename}{圖}
% \renewcommand{\tablename}{表}
% \renewcommand{\abstractname}{摘要}
% \renewcommand{\contentsname}{目錄}
% \renewcommand{\listtablename}{表格目錄}
%\renewcommand*{\bibfont}{\footnotesize}
\titleformat*{\section}{\Large \bfseries \sffamily}
\titleformat*{\subsection}{\large \bfseries \sffamily}
\titleformat*{\subsubsection}{\bfseries \sffamily}
\setcounter{tocdepth}{2}
\setcounter{biburlucpenalty}{9900}
\setcounter{biburllcpenalty}{9900}
\setcounter{biburlnumpenalty}{9900}

\title{\sffamily{\small MID Working Thesis} \\ \bfseries Pitfalls and alternatives of foundation models}
\author{Ben Po-Sheng Cheng\thanks{Rhode Island School of Design. Contact: pcheng01@risd.edu}}
\date{\today}
\begin{document}
\pagenumbering{roman}
\maketitle
\begin{abstract}
    Ever since OpenAI unveiled ChatGPT to the public, the idea of Artificial General Intelligence (AGI) has stormed the tech industry with its promise to be ``the most impactful technology in human history''. (\cite{sama}) This is part of a larger paradigm shift of AI development towards general capability Foundation Models that consumes humongous resources. The conversational chatbot interface of ChatGPT has become a norm of how people expect to interact with AI. However, the paradigm of foundation model is dangerous. Enormous amount of energy, labor and capital have been poured into the development of foundation model despite its several overlooked issues, like reinforced bias and inequalities, unhealthy distortion of the science community and the environmental cost of datacenters. 

    The goal of this work is to challenge this AI paradigm by resurfacing its messy logistics and revealing other potential alternatives through devices that embed this information into the way people interact with AI products. I wish to spark discussions on whether we should be so devoted to this path and provide a pointer towards other possible trajectories for AI's future that are more sustainable, ethical, and humane. 
\end{abstract}
%\subsection*{Ben Cheng}
% Artificial intelligence, AI, is often touted as the most promising technology advancement in recent years. Sam Altman, OpenAI's founder, CEO and one of the most prominent voice in AI development, often refers to Artificial General Intelligence (AGI) as ``the most impactful technology in human history'', (\cite{sama}) stating its potential to solve cancer and climate change. Anthropic, another big player in the Large Language Model (LLM) market, recently finished a 13 billion funding round. (\cite{anthropic})

% These promises and phenomena have resulted in AI attracting a lot of attention, media, capital, and talent. Consequently, the ``unseen sacrifices'' of the recent development in AI are severely overlooked. When people think of AI, they often think of the LLM-based chatbot and the positive capabilities it brings about to their daily lives. However, AI, or more generally machine learning, has so many more forms and application that are worth exploring. What we are observing now is all the resources being devoted to one form of it while its ``hidden cost'' (externalities) being ignored.

% I wish to ask ``how can we shift people's focus of AI development with designed objects?'' through designing AI-based objects,  products or interaction schema to reveal the hidden sacrifices or unseen alternatives of AI. The current AI paradigm, largely based on OpenAI's scaling law, (\cite{scaling}) extracts extremely large amount of resources to train huge models that might just keep repeating its training materials. (\cite{parrots}) A transformer-based LLM in 2019 can easily create 5 times the carbon footprint than a car throughout its lifetime.(\cite{strubell-etal-2019-energy}) In order to build datacenters large enough to host these models, water are being extracted in villages suffering from drought (\cite{empire}) and labors in developing countries are exposed to sensitive contents for very long working hours to moderate the output of these models. (\cite{empire}) LLM developers extracts and scraps online content to create lucrative models and sanitize them as free, public or open-sourced ``texts''. (\cite{scraping}) All of these ``unseen sacrifices'' are hidden from the smooth and friendly chatbot products made by these companies and I wish to explore ways of designing products that resurfaces these externalities.

% The ultimate goal of raising people's awareness of AI is to redirect attention and resources to the other various forms of AI that have the same potential to be beneficial. For example, neuro-symbolic AI has been seen as an alternative branch of AI that utilizes human-programmed causalities to make decisions rather than purely rely on statistical correlation. (\cite{reboot}) Smaller, more specialized models have also been used to save and reproduce disappearing languages. (\cite{maori}) (On the other hand, LLMs tend to reinforce existing biases and accelerate the vanishing of languages. (\cite{empire})) Researchers have also been advocating regulations and systems to make the energy consumption of LLMs more visible to consumers. (\cite{Luccioni_Gamazaychikov_Hooker_Pierrard_Strubell_Jernite_Wu_2024b})

% I believe in the benefit of AI as a technology advancement, but I also believe it should a open and diversified process, just like every progress made in science. By designing objects that spark conversations and stimulate new perspectives, I wish the discussion and development of AI can serve more than the monetization of LLMs.

% Ever since OpenAI released ChatGPT to the public, it has stormed the tech industry with its promise to be ``the most impactful technology in human history''. (\cite{sama}) Based on the scaling law invented by OpenAI, (\cite{scaling}) ChatGPT has shifted the paradigm of artificial intelligence towards Large Language Model (LLM), which requires a substantial input of resources to develop. The friendly and helpful chatbot interface of ChatGPT has became what the public understand as AI. However, it has been reported that the development of LLMs consumes an astronomical amount of capital (\cite{anthropic}), energy (\cite{strubell-etal-2019-energy}), labor (\cite{empire}), water (\cite{empire}) and scraping of online content (\cite{scraping}).

% I wish to design objects, artifacts and interaction schema to resurface these messy logistics behind the current AI paradigm. They will embed unexpected disruptions into what people expect to be the familiar and friendly interfaces of AI products. The goal is to question the current prominent paradigm of AI and to reveal other possible alternatives. For example, AI systems that utilizes human-programmed causalities to make decisions rather than purely rely on statistical correlation, (\cite{reboot}, \cite{parrots}) smaller, more specialized models geared towards helping cultural minorities (\cite{maori}), and regulation mechanisms. (\cite{Luccioni_Gamazaychikov_Hooker_Pierrard_Strubell_Jernite_Wu_2024b})
\newpage

\tableofcontents
\newpage
\pagenumbering{arabic}
\section{Introduction}
Foundation Models denote the paradigm shift undergoing in the Artificial Intelligence (AI) research and development discipline. It refers to developing models that are large (in terms of number of parameters/weights of neural networks), possess general capability, and act as a foundation or intermediate for downstream AI products. (\cite{bommasani2022opportunitiesrisksfoundationmodels}) Their large scale means they are extremely expensive to develop, and only a handful of resourceful companies are able to afford the computing power, energy, water, and labor required. Their general capability means their behavior is unpredictable, and there is a pressing lack of proper benchmarks to evaluate their performance and externalities. They have a broad impact and implications as they are adapted by all kinds of AI applications; however, their intrinsic biases are also inherited by every downstream product, reinforcing and amplifying existing stereotypes and inequalities.

\subsection[Issues and Alternatives]{Issues and Alternatives of Foundation Model}
This work starts by examining these issues of the current AI paradigm. The extraction of resources is discussed first. Specifically, the energy consumption of datacenters used to train and host foundation models is rising exponentially. Water used to cool down the computing units inside datacenters is also adding to the environmental instability of those areas already suffering from extreme weather. Chatting with a generative AI model seems like a harmless small action; however, actual, physical machines are running and consuming resources to produce responses.

As models grow larger and larger, datacenters are also growing bigger. A modern datacenter for AI training can easily hit power consumption in the magnitudes of megawatts, comparable to the entire New York City Subway. They also take physical land from precious nature environment or indigenous communities. They wipe out landscape, create noise and does not leave job opportunities for the locals.

The training datasets for these models also have to grow bigger. For example, to train its 175 billion parameter model GPT-3, OpenAI used the Common Crawl, which is basically the entire public internet, as part of the training dataset. These datasets contain sensitive content that was not moderated before being fed into the training of the model. It is reported that, in order to create the guardrails preventing models from generating sexual and violent content, foundation model developers like OpenAI indirectly hire people from the global south to moderate the output of their models. They are exposed to large amounts of text describing child sexual abuse, sexual slavery, and more while receiving minimum mental health support.

The overrepresentation, misrepresentation, and underrepresentation across cultural, linguistic, geographical, and racial groups in the datasets also dictate these models' behavior and capability. Countless research has shown that models exhibit dialectic prejudice and visual biases. Performance disparities also mean overrepresented groups are more likely to benefit from using AI as tools, further amplifying existing inequalities.

The central issue that motivated this work is the consolidation of power in the AI field. Due to the absurd cost required to develop foundation models, it has increasingly become difficult for independent researchers in academic institutions to conduct state-of-the-art research. Independent verification and benchmarking of the privately-developed models are also extremely costly. If AI really is what we imagine as the frontier of scientific advancement, it should be developed with the mindset of knowledge-sharing and peer-review. However, developers like OpenAI or Anthropic stopped open-sourcing their models, treating them as proprietary products. They also use sponsorships and funding to distort AI Ethics research, creating challenges in conflicts of interest and academic integrity. 

None of these would have posed such a threat to our trust in AI if the foundation model was not the only paradigm in AI development. Nevertheless, we are observing homogenization in the research and development of machine learning systems. Almost all of the cutting-edge language models are now based on a very similar Transformer architecture. Other fields in machine learning that used to use different model architectures also started to adopt the same paradigm. Combined with the huge influence for-profit AI developers now have, a picture of autocracy starts to emerge.

This is why exploring alternative paradigms of AI development is crucial. A lot of the issues of foundation models come from their enormous size, so different ways to develop smaller models should be explored. Researchers should try to develop models grounded in real-world usage instead of pursuing hypothetical benchmarks. A deeper understanding of real-world applications means the models can be tailored to specific usages, potentially creating more computing-efficient models. Instead of pursuing one single ``master algorithm'' to tackle every task, distinct model architectures should be used because every task has a different internal logic.

If the goal of AI is to create computing systems that match the capability of human intelligence, we should try to learn from the human mind. Our brain utilizes various different senses to aid decision-making, but computing systems still only have a very limited capability to sense the physical world. For example, computers have no sense of tactility and smell. The human mind also possesses what is generally referred to as common sense, which can be characterized as a general understanding of real-world causality. Foundation models, however, can only learn the most surface level of common sense even after being trained on a large corpus of text data. We should look into ways to design models that are able to utilize the common sense of humans. Similarly, foundation models try to learn everything about the world but do not have any internal representation, unlike the human mind which internalizes experiences into our desires, preferences, and beliefs. These hints from the human mind are all inspirations for models that can perform better in the real world without having to be so humongous.

\subsection{Outcome and Methodology}
The goal of this work is to spark conversations, or to be the conversation itself, around the topic of whether we should be devoted to the paradigm of foundation models as the one single path for the future of AI. I believe that the hidden cost and potential alternatives of foundation models are a severely overlooked topic in the public. My wish is to use designed objects as a different way to discuss the issues and alternatives of foundation models as the paradigm of AI. 

These objects also represent novel interfaces for human-AI interaction that have not yet been explored. Some of them try to indirectly raise users' awareness of the unseen ``physicality'' of AI -- namely the physical servers consuming energy, generating heat, and making noise. Others try to explore what a human-AI interface could look like for the different alternative paradigms discussed in this work. The goal here is to use the language of objects as a medium for debate, discourse, and discussion. The objects do not directly represent feasible products or AI technology. Instead, they are tools for communicating ideas. 

% \subsection[Audiences]{(Potential) Audiences\footnotemark}
% \footnotetext{Work in progress. This section identifies the potential audiences for this work. As the direction of this work develops the target audience will be narrowed down.}
% Due to the critical and discursive nature of this work, it is important to note that whoever the target audience is, they are not considered as the ``user'' of the product of this work in a traditional sense. The overarching goal of this work is to sparkle discussions and to be the discussion itself. The so called audiences are hence considered as the spectator of the designed artifacts. The user of a product decides the context and setting of using it, however, the spectator of a product does not decides how it is situated -- they encounter and interact with it in a context defined by the author. Below I present potential audiences, along with the context in which I intend to show the work to them. They are general users of generative AI products, academic reseachers and museums/shows.

% \subsubsection{General User of AI}
% According to \cite{NBERw34255}, over 10\% of global population uses ChatGPT every week. Over the past year, use of ChatGPT for non-work related tasks has grown from 47\% to 73\%. Using derived product of foundation model like ChatGPT has become an integral part people's daily life. How these users uses AI ultimately influences how AI is developed. At a macro level, topics user seeks help from AI differs a lot across different products (\cite{NBERw34255,appelmccrorytamkin2025geoapi}). It is an worthwhile effort for me to study how people uses AI and their experiences of using AI in order to come up with an effective way to integrate disruptions to their way of interacting with AI.

% \subsubsection{Academic Researchers}
% As an emerging technology, most of the cutting-edge development made in AI was born in research labs. The paradigm shift of foundation model has changed that. However, energy of reseach in the critique and evaluation of AI has not cease to grow. My goal for the product of this work is to enable a conversation on the pitfalls and alternatives of the current AI paradigm, and people in the academic world have been having this conversation for a long time. They have conversations by publishing papers and attending conferences. This work adds to their conversation with physical, tangible interactions that enables them to see the their own work through a different lens. I believe conferences like ACM FAccT\footnotemark or AIES\footnotemark could be a ideal occasion for this work to be situated. Understanding the language of the academic world and extrapolating/visualizing the real-world implication of AI research would be critical to engage this audience.
% \footnotetext{ACM Conference on Fairness, Accountability and Transparency}
% \footnotetext{AAAI/ACM Conference on AI, Ethics and Society}

% \subsubsection{Museums}
% Museums or shows usually provide the most flexibility for a work to be shown. They are ultimately places that allow for engagement with the general public in ways exactly as intended by the author. However, making a work for these settings means a certain level of spectacular elements needs to be implemented very well. Compared to designing for the general user of AI, which focuses more on the interactive elements, designing for museums/shows might require more elaborate performative layers. The focus might thus shift towards finding clever exaggerations, contrast, asymmetry or chaos.

\section{Relevance}
Built upon the grounds of AI Ethics research, this work aims to communicate the findings in this area in an interactive and tangible manner that speaks to the general public. The research of this work is largely divided into two parts, the issues of the current AI paradigm, and the alternatives. The development of AI is an ongoing incident that influences contemporary society in many different ways, as admirable efforts of researchers in AI Ethics has already shown. The multifaceted impact of the current AI paradigm are creating, amplifying and reinforcing societal issues that I believe people are not paying enough attention to but are dangerous in the long term.

Constructive criticism can be a vital force in challenging the current AI paradigm. Presenting what could be done or could have been done in the development of AI adds to the conversation of whether we should be so devoted to foundation models. AI and machine learning is a scientific field that has existed for almost 60 years. It has taken many different forms and propositions; the foundation model is a fairly recent breakthrough in comparison. There are still many different paradigms and perspectives on how AI/ML should be developed, and revisiting these ideas is a worthwhile effort to argue that we should be investing more equally across different paradigms rather than focusing on just one.

The rise of foundation models as the main AI paradigm is a pressing and current issue. Over 10\% of global population uses ChatGPT weekly. (\cite{NBERw34255}) Nearly 10\% of US businesses has adopted AI in their workflow. (\cite{appelmccrorytamkin2025geoapi}) As you are reading this paper, datacenters across the globe are consuming energy, water and labor resources at an unprecedented scale. Below, a brief outline of the various issues created by foundation models are presented.

\subsection{Fundamentals}
{\itshape TODO: This section talks about technical foundation of the current AI paradigm. The focus will be on the intrinsic downfalls implied by these internal properties.}
\subsection{Energy and Water}
Energy usage is one of the most literal and direct implication of foundation models. Based on OpenAI's scaling law, (\cite{scaling}) the uprise of foundation models is fueled by enormous amount of training data and huge number of the parameters inside the model. Behind this paradigm, its seemingly impressive capabilities are made possible by an extremely large amount of computing power. The training of a state-of-the-art large language model require tens of thousands of computing units running non-stop for weeks. Each computing unit can consume up to 300 watts of power when running at full speed. They also generate a lot of heat, which takes additional energy and water to dissipate. Power consumption of the largest modern datacenters are designed to consume around 150 megawatts, equivalent to nearly 122,000 american households.

As estimated by \cite{strubell-etal-2019-energy}, the training of a large language model can generate more than 5 times the carbon footprint of the lifetime of a car including fuel. Besides training, running these models is also a energy hungry task. According to \cite{10.1145/3630106.3658542}, the energy required to generate an image using large foundation models is equivalent to one fifth of the energy stored in an iPhone battery. Energy consumption of these datacenters is only predicted to continue rising, surpassing that of Japan (1000 TWh) in 2026. (\cite{iea})

Humongous datacenters not only take electricity to run, they also require water for cooling down the computing units. \cite{10.1145/3724499} estimated that the training of OpenAI's 175 billion parameter model GPT-3 consumed 5.4 million liters of water in Microsoft's datacenters. Depending on where the model is hosted, it also consumes 500ml of water for every 10-50 medium-length responses generated.

\subsection{Labor and Environmental}
As big techs race to spend tens of billions building out the biggest datacenters to support the foundation model boom, land, communities and nature environments are being exploited. On the outskirts of Santiago, Chile, the municipality of Quilicura hosts various factories and Google's datacenter. Only two decades ago, Quilicura was a rural town home to the rich biodiversity of the Atacama Desert. Now, there is only industrial complexes and the ``Quilicura Urban Forest'', built by Google as a community giveback in 2019. However, there are not many residents still living there as it has become an industrial area. Like most datacenters around  the world, Google's datacenter also does not provide any job for the locals except for the temporary construction work -- in 2024, a job posting for mechanical technician for the datacenter requires applicant to submit resume only in English. (\cite{empire})

Generative foundation models like GPT are trained on enormous amounts of datasets, many of which are impossible to filter or censor. As a result, the original models produce sensitive content, and companies like OpenAI hire Silicon Valley middlemen like Sama or Scale to access cheap labor in developing countries to moderate the generated content. It is reported that Sama hired people in Kenya for an average of \$1.46 to \$3.74 to do this work. They are exposed to sexual, violent text content including child sexual abuse, slavery, and bestiality while their employer provides minimum mental health support, which often results in serious mental health conditions. (\cite{empire})

These examples highlight the multifaceted impact of western technological advancements. They exploit resources from the underprivileged and destroy their land, community, and lifestyle.

\subsection{Bias and Ripoff}
Foundation models act as a general intermediate for various AI products; they are often fine-tuned, augmented, or adapted for more specific use cases. Because they are trained with extremely large datasets like the Common Crawl, the bias of the datasets is inherited by the models. For example, there is more content on the internet in English than in any other language. The large amount of online content that is hateful, abusive, and toxic inevitably becomes part of the training material and also directs the model's behavior. The overrepresentation, underrepresentation, and misrepresentation across cultural, geographical, and racial groups are also carried through to foundation models.

These intrinsic biases are inherited by downstream applications that adapt foundation models, creating widespread performance disparities for various use cases. For example, \cite{doi:10.1073/pnas.1915768117} looked into how African American English speakers who cannot access reliable speech recognition technologies are in a disadvantageous position when these technologies are used in job interviews or transcribing courtroom proceedings. According to \cite{Hofmann2024}, models exhibit stereotypes about speakers of African American English. They may be more likely to suggest that speakers of African American English be assigned less-prestigious jobs, be convicted of crimes, and be sentenced to death. \cite{10.1145/3442188.3445928} found that models fail to detect faces of people with darker skin tones.

On top of that, the training data are often obtained without clear consent. Several news outlets have filed lawsuits against foundation model developers like OpenAI and/or Anthropic for illegally using their publications as training materials. These materials are usually consumed by the model training to update its parameters. They only exist in the generic form of numbers and vectors, erasing the identity and intention of the original authors. Image generation models are trained on images created by artists and illustrators, only to eventually replace them in the workplace.

\subsection{Economic Inequality}\label{econ}
It's easy to describe AI as a groundbreaking technological advancement that benefits all of humanity, solving climate change, cancer, or other troubling issues that pose a significant danger to our species. However, none of these promises seem remotely close to being realized, and the actual economic impact of AI paints a much grimmer picture.

\cite{NBERw32487} estimates the Total Factor Productivity\footnote{In simpler terms, the combined productivity of production input (capital and labor). Traditionally its growth is attributed to technological advancement.} growth generated by AI to be 0.66\% over a ten year period, which is much more modest than people might expect. Moreover, as estimated by \cite{690378712}, emerging technologies like social media may have negative impact on social welfare even if they create economy growth on paper. We've already seen various malicious and manipulative uses enabled or exacerbated by AI, like disinformation. (\cite{csetbuchanan})

According to \cite{10.1093/cjres/rsz022}, when new technologies are able to drive substantial productivity growth, even if a large portion of routine tasks are automated and thus lower-skill workers are replaced, the reorganization of production processes creates new tasks and jobs for human labor. However, if the productivity growth is modest or if the new technology's sole focus is to replace human labor, new tasks may not emerge, and task displacement means low-skilled workers will suffer from job loss. This is what Acemoglu described as ``so-so'' automation, and AI, with its estimated 0.66\% productivity growth, seems dangerously close to falling into this group.

Recent developments in economics suggest that technology, in particular those that enable automation in the production process of goods or services, creates an imbalance and unequal outcome for people in different income or demographic groups. (\cite{https://doi.org/10.3982/ECTA19815, https://doi.org/10.3982/ECTA19417}) Automation usually replaces low-skilled labor in the production process. In other words, the people who lose their jobs to AI are theoretically those who are already in the lower income bracket. Additionally, more automation means a higher return on capital investment, delivering more earnings to the wealthy. \cite{SKARE2024102719} shows that the capital stock of AI is unevenly distributed and positively correlated with exacerbated income inequality. Using online job posting data, \cite{ssrn4874061} finds that hiring for lower-level office jobs declined after ChatGPT's release, while demand for higher-skilled workers increased.

Economic inequality not only exists across income groups but also across demographics. Preliminary estimation in \cite{NBERw32487} suggests that AI might cause the wages of low-education women to decline. \cite{appelmccrorytamkin2025geoapi} finds that countries with higher AI usage are concentrated in North America, Western Europe, and Northeast Asia, while the global south almost entirely falls into the minimum usage group. In their theoretical framework, \cite{doi:10.1086/737233} found that AI is more likely to displace workers from complex problem-solving work to routine tasks in a developing economy, unlike in an advanced economy where people will be better positioned to use AI to help them tackle more complicated work. All of this seems to suggest that the prevalence of AI is empowering the already-empowered while underprivileged groups will benefit less from AI.

Rising economic inequality has significant implications. Long-lasting stagnant wages and loss of employment opportunities in certain groups create popular discontent that sometimes fuels disruptive populist movements. Uneven distribution of income also leads to the consolidation of wealth for certain individuals or business sectors, which in turn results in the consolidation of power. This is particularly true in the case of the current AI landscape, in which very few people possess immense power over the trajectory of its development and are able to steer the future of AI to benefit themselves. 

\subsection{Power Concentration}
{\itshape TODO: Add description on how the trajectory of AI is dictated by very few powerful entities/men. OpenAI's power struggle. Capital investment. Big techs are the only player. Effective Altruism}
\subsubsection{Death of Open Science}
The paradigm of foundation models represents the homogenization in the field of machine learning research and development. Specifically, \cite{bommasani2022opportunitiesrisksfoundationmodels} characterize homogenization as ``the consolidation of methodologies for building machine learning systems across a wide range of applications.'' The origin of foundation models stems from the field of Natural Language Processing (NLP), with key breakthroughs including self-supervised learning, word embedding, the transformer (\cite{vaswani2023attentionneed}), and the bi-directional encoder-decoder (\cite{devlin2019bertpretrainingdeepbidirectional}). These technical architectures represent the essential properties of foundation models, laying out their uncanny capabilities but also their intrinsic downfalls.

Today, almost all state-of-the-art NLP models are derived from a handful of foundation models like the bidirectional encoder model BERT. (\cite{bommasani2022opportunitiesrisksfoundationmodels}) The transformer architecture of large language models is also beginning to be widely applied to other areas like speech recognition, image, biology, and reinforcement learning. The phenomenon of foundation models as a paradigm taking over other forms of machine learning systems is the key motivation of this work, particularly because of the ongoing issues it's creating and the huge amount of resources required to develop them. AI, as a frontier discipline of science, should be explored openly. However, we are observing foundation models crowding out the resources for research and development in other machine learning systems. 

Even worse, the development of foundation models is so expensive that only a handful of ``big-techs'' can afford it now. It's getting more and more difficult for independent researchers unaffiliated with big-tech companies to conduct research in cutting-edge foundation models. Because of the sheer size of these models, their performance claimed by the private developers is extremely hard to verify independently. Evaluating their real-world cost and externalities has become equally difficult for the same reason.

Founded as a non-profit to ensure the safety of AI development, OpenAI stopped open-sourcing its models after GPT-3 in 2020, citing security reasons. This has become the norm across many big players in the field, including Google and Anthropic. (\cite{empire}) Subsequently, OpenAI has transformed itself into a private for-profit company. The field of cutting-edge foundation model development increasingly looks like another technology race in Silicon Valley, rather than an open community in which members share knowledge and breakthroughs with each other for the advancement of science that benefits humanity. If AI is what we believe to be the next leap forward in human civilization, the path we are taking to get there seems like an extremely dangerous one.

The thriving community of AI Ethics researchers set out to tackle these threats by collaborating on research evaluating the issues of foundation models and coming up with appropriate benchmarks. Nonetheless, big-techs still find their way to try to consolidate their power in this community. \cite{10.1145/3461702.3462563} finds that big-techs use the same strategy as big tobacco in the 1950s to undermine research questioning their products. They publicly claim their emphasis on and care for AI safety and even set up internal research divisions to conduct research. They also sponsor and provide funding for top conferences, events, and individuals in AI Ethics. This presents conflicts of interest and threats to academic integrity. According to the cited work, there are more researchers in AI Ethics at top institutions that have received funding from or been affiliated with big-techs than those who have not. In 2020, researcher Timnit Gebru was ousted by Google over the publication of her now-famous paper Stochastic Parrots, highlighting the issues of large language models sheerly repeating their own training data. (\cite{empire}, \cite{parrots})
% \subsection*{Literature Review\footnotemark}
% \footnotetext{To be dissolved. This section is a temporary index for organizing reference materials.}
% \subsubsection*{Technical Background}
% \begin{itemize}
%     \item Scaling law (\cite{scaling})
%     \item Word embedding (\cite{mikolov2013efficientestimationwordrepresentations})
%     \item Transformer (\cite{vaswani2023attentionneed})
%     \item BERT (\cite{devlin2019bertpretrainingdeepbidirectional})
%     \item Visual (\cite{radford2021learningtransferablevisualmodels}, \cite{dosovitskiy2021imageworth16x16words}, \cite{ramesh2022hierarchicaltextconditionalimagegeneration})
%     \item Foundation Model (\cite{bommasani2022opportunitiesrisksfoundationmodels})
% \end{itemize}
% \subsubsection*{Use Cases}
% \begin{itemize}
%     \item Business (\cite{appelmccrorytamkin2025geoapi})
%     \item Consumer (\cite{NBERw34255}, \cite{NBERw32966})
% \end{itemize}
% \subsubsection*{Impact and Pitfalls}
% \begin{itemize}
%     \item Water, labour (\cite{empire}, \cite{atlas}, \cite{10.1145/3724499})
%     \item Stochasticity, scale (\cite{parrots}, \cite{10.1145/3715275.3732006}, \cite{thompson2022computationallimitsdeeplearning})
%     \item Energy (\cite{strubell-etal-2019-energy}, \cite{10.1145/3381831}, \cite{10.1145/3630106.3658542}, \cite{iea})
%     \item Bias, inequalities, ripoff (\cite{scraping}, \cite{Hofmann2024}, \cite{zhou2021frequencybaseddistortionscontextualizedword}, \cite{doi:10.1073/pnas.1915768117}, \cite{10.1145/3442188.3445928})
%     \item Power concentration, academic distortion, death of open science (\cite{10.1145/3715275.3732006},  \cite{abdalla-etal-2023-elephant})
%     \item Disinformation (\cite{csetbuchanan})
%     \item Economics (\cite{NBERw32487}, \cite{SKARE2024102719})
%     \item STS (\cite{10.1145/3351095.3372871}, \cite{kulveit2025gradualdisempowermentsystemicexistential})
% \end{itemize}
% \subsubsection*{Alternatives}
% \begin{itemize}
%     \item On-line learning
%     \item Regulation (\cite{doi:10.1126/science.adp1848}, \cite{Luccioni_Gamazaychikov_Hooker_Pierrard_Strubell_Jernite_Wu_2024b})
%     \item Research-driven development (\cite{10.1145/3461702.3462563})
%     \item Cognitive AI (\cite{reboot})
%         \begin{itemize}
%             \item World model (\cite{bruce2024geniegenerativeinteractiveenvironments}, \cite{gupta2022maskvitmaskedvisualpretraining})
%         \end{itemize}
%     \item Application-driven ML (\cite{10.5555/3042573.3042809})
%     \item Community-participatory ML
% \end{itemize}

\section{Alternatives}
Known as the ``father of AI'', congnitive and computer scientist Marvin Minsky wrote in his book \textit{The Society of Mind}, ``The power of intelligence stems from our vast diversity, not from any single, perfect principle.'' He portraits the human mind as a society of simple but very different cognitive processes, also known as agents. (\cite{societyofmind}) Each of these individual thinking entity has different internal logics, combined to form the vast intelligence of our human mind.

The current paradigm in Artificial Intelligence, however, is devoted to building systems that are the complete opposite. The paradigm of foundation model suggests that a single, humongous ``master algorithm'' will be able to reproduce the extremely complicated and inter-weaved cognitive capabilities of the human mind. In particular, the uprise of Large Language Models implies that language is the only necessary tool for intelligent systems to reason and make decisions. Instead of incorporating a fundamentally different architecture for vision capabilities, the recent development of multi-modal Vision-Language Models uses the same word embedding techniques for understanding images, essentially treating each small ``patch'' of an image as a word. (\cite{dosovitskiy2021imageworth16x16words}) It seems like the world firmly believes that this one single way of developing AI systems can lead to something on par with the human mind.

The core proposition of this thesis is that instead of pursuing an omni-capable master algorithm, resources should be more equally devoted to developing various kinds of AI systems. This section outlines the other alternative paradigms of AI that I believe should have been given the same amount of attention. Some of them draw inspiration from the intrinsic downfalls of foundation models, others from the negative externalities we are observing around the world, and still others from the cognitive science that depicts how human intelligence works.

\subsection{Application-Driven AI}
The premise of foundation models, or more specifically transformer-based large language models, as the major breakthrough towards artificial general intelligence is that one giant model will be sufficient to encapsulate all sorts of human intelligence. Indeed, in recent years, we've seen substantial improvement in the performance of these models. However, The so-called ``performance'' are usually evaluated with hypothetical datasets and metrics that doesn't translate to real-world capabilities.

This is what Kiri Wagstaff of the California Institute of Technology describes as ``machine learning for machine learning's sake''. (\cite{10.5555/3042573.3042809}) A lot of these improvements don't necessarily translate to real-world usages. For example, a computer vision model might be able to successfully detect pedestrians with a 99.9\% rate on a benchmark dataset; nevertheless, failing to identify one pedestrian in every 1000 is still fatal and doesn't mean the model is nearly close to being capable enough to be applied to critical applications like autonomous vehicles. The long tradition of machine learning is based on a stochastic process instead of deterministic decision-making. Much of artificial intelligence's vast improvement in performance is driven by raising the probability of success, not real-world usage. Moreover, the same metric for success rate is not comparable across different applications. 99.9\% might not be enough for self-driving cars but might be sufficient for identifying food ingredients. This is again to say that the pursuit of performance in benchmarking metrics tells us nothing about how usable it is in real-world applications.

Additionally, Wagstaff also argues that the machine learning community needs to engage with real-world domain experts more closely. The prevailing paradigm of collaboration with other domain experts usually only happens for curating and annotating training datasets (\cite{10.1145/3411764.3445518}), but little for creating the appropriate evaluations for real-world applications and for their potential impact.

Creating models that have real-world applications in mind from the initiation of their development also boasts the benefit of having smaller, more efficient models. Over the past couple of years, we've seen developments in machine vision, audio, and robotics all converge to the paradigm of a multi-modal transformer architecture. The one-model-to-rule-them-all concept does seem tempting, but the truth is that different tasks may require different architectures for more efficient computing. Even having more application-specific transformers may also be beneficial.

Smaller models also enable the possibility of being run locally on smaller hardware, which allows their users to have a very different relationship with the models. Models that can be run locally will also be easier to be trained and fine-tuned by their users, which means they can adapt more granularly to a user's personal needs and lifestyle.

\subsection{Cognitive Algorithm}\label{cognition}
If the goal of Artificial Intelligence is to build computing systems that rival human intelligence, it makes sense to mimic how the human mind works. Unfortunately, the human brain and mind are such complicated systems that to this date, we still haven't solved their mystery. However, psychology, cognitive and neural science are still able to give us some hints.

New York University professors Gary Marcus and Ernest Davis wrote in their book \textit{Rebooting AI} that drawing from Kant's \textit{Critique of Pure Reason}, ``time, space, and causality are the three philosophical grounds fundamental to the human mind.'' (\cite{reboot}) When seeing a car coming while trying to cross the road, humans are able, in almost no time, to estimate how far the car is, how long it's going to take for it to be dangerously near, and how long it would take to cross the road, and then make a decision. Such simple everyday tasks are usually comprised of many complicated thoughts. We are able to ``sense'' the distance and time of the car with little to no brain effort, at the same time considering other factors, like the sound volume of the coming car or whether you are about to be late or not.

\subsubsection{Purely Analog Senses}
Given a well-defined context and sufficient knowledge, existing machine learning systems might be able to understand the time duration of a certain event, but they are only able to do so by inducing from mathematical formulas. They have the knowledge of time, but humans have a sense of time. Similarly, AI has an understanding of the shapes of objects, but it might not be able to immediately induce their functionality from the shapes. Imagine seeing a cheese grater for the first time; we are able to reason with its scale, shape, material, and texture to immediately guess what it might be for and how to use it. This kind of holistic reasoning combining our innate understanding of time, space, and causality is exactly the kind that intelligent machines should learn from the human mind. Of course, by showing a cheese grater and its text description to a model during training, current machine learning systems would have no problem understanding what a cheese grater is. But perhaps that's why foundation models have to be large models consuming almost all the available existing data to train. It is important to note that fundamentally, current state-of-the-art models are still language models that understand everything through language and language only. Computers fundamentally cannot ``sense'' anything but can understand its ``abstraction'' as language or other symbolic systems, like a mathematical formula.

A maybe more efficient way to build AI systems is to look into the other ``senses'' that do not yet exist in the digital world so that AI can reason and induce more holistically. Upon seeing the picture of a textile, we are immediately able to guess its texture, flexibility, or even whether it's waterproof or not, all without touching it. The existing ``senses'' computers have include light, sound, force, temperature, electrical potential, magnetic fields, and more. All of these are mathematically well-defined physical properties, and they all have corresponding sensors that can transform them into digital readouts. We humans, however, do not rely on the understanding of physical properties to perceive the world. We understand the world through smell, texture, haptics, and more. Machine learning systems, on the other hand, do not possess this kind of capability to understand these ``mathematically ill-defined'' and thus purely analog properties. 

Some recent developments in the field of the so-called world model claim to be able to generate realistic simulations of the physical world. (\cite{bruce2024geniegenerativeinteractiveenvironments,gupta2022maskvitmaskedvisualpretraining}) What these models actually are is models capable of playing video games. It is unsure whether they are just reproducing common physical phenomena in video training data or if they indeed have an understanding of the rules of physics. Even if they do, their output is still limited to just video or visual images -- far from all the complicated analog senses humans have. We should tap into these unknown territories of the digital world, trying to come up with more abstraction or symbolic systems to teach machines more nuances of this physical world. This can potentially eliminate the vast resources required to teach models everything with empirical training materials. And in the context of the current paradigm of large language foundation models, giving computing devices more ``senses'' to characterize the world also saves us the effort and data size of trying to describe everything in text.  

\subsubsection{Common Senses and Causality}
When we say Novak Djokovic has been playing professional tennis for 20 years, we don't mean that he has been playing tennis non-stop for the continuous time duration of 20 years. The human mind is capable of understanding the implied meaning that tennis has been Novak Djokovic's profession for 20 years. However, a language model that is trained to understand the literal meaning of texts might not be able to decipher the latent implication.\footnote{I prompted Claude Sonnet 4.5, ``suppose one year is 365 days, one day is 24 hours, one hour is 60 minutes. Novak Djokovic has been playing tennis for 20 years, how long has he been playing tennis?'', and then the chatbot went on to actually convert 20 years to hours and minutes. Although it is also quite astonishing that the model was able to understand the implied meaning of the first part of the prompt, a human being would probably first ask for clarification.} This is an example of what we vaguely describe as ``common sense''. We, as humans, do not need to be taught that it is impossible to continuously play tennis for 20 years, even for a professional athlete, and are able to comprehend the idea of a profession because we are able to understand the world and human lives by employing different knowledge frameworks simultaneously. Instead of reading 20 years as 365 days and one day as 24 hours, our sense of 20 years as ``a very long time'' substantially outweighs the symbolic meaning of a year as a unit of time. The human mind is firstly a sensing entity and then an abstraction decoder. We built the various symbolic systems based on our innate senses; machine learning systems, however, can only induce from the abstraction and symbols. As a result, the lack of common sense has been a challenge for AI.

An alternative AI paradigm is trying to program common sense, or the general understanding of causality, into purely mathematical models. We understand the basic causal relationship of everyday events and objects, but even the largest foundation model today does not have any mechanism to employ this kind of knowledge. It only has the most surface-level understanding of our common sense even after being trained on the largest corpus of text material. Carnegie Mellon University computer scientist Douglas Lenat was an advocate for intelligent machine systems with programmed causality. (\cite{LENAT1984269}) His long-term project Cyc aims to create a symbolic system and a knowledge base database that maps out the relationship of everyday common sense and implicit knowledge. It's a decision-making algorithm that is based on purely deterministic and pre-programmed relationships. After almost 40 years of employing philosophers and programmers to write down rules of the world for machines to understand, the project unfortunately never gained much traction. However, it is a sharp contrast to the strictly stochastic and training-based paradigm of foundation models. 

Amidst the enormous resources being poured into developing large models, maybe it is a good time to revisit the idea that machines should be able to employ some simple pre-programmed rules that describe how the world works, instead of trying to make the algorithms learn every single relationship embedded in our common sense. 

\subsubsection{Learning versus Innateness}
Large foundation models, as statistical pattern-matching machines, try to generalize everything into a widely applicable algorithm. Even if they were able to learn the entire common sense of the human mind from a large corpus of data, not everything follows rules. Our human intelligence makes decisions based on not only the abstraction and generalization of our various senses and experiences but also our innate values. We internalize our experiences into an individual's unique logic.

Psychology provides us with plenty of evidence that human decision-making does not only come from learnings from the outside world but also from internalized values. Psychologist John Watson of Johns Hopkins University, a major pioneer in \textit{behaviorism} psychology in the early 1900s, famously claimed that a child's behavioral pattern can be dictated purely by controlling every factor of the environment in which the child was raised. (\cite{behaviorism}) Only a couple of decades later, Noam Chomsky, who is usually seen as the founder of \textit{cognitive} psychology, published the groundbreaking work in linguistics \textit{Syntactic Structures}. (\cite{noamchomskylinguistic}). Along with his other works, he characterizes language understanding not as pattern-matching of past encounters of sentences but as a result of internalized grammar. (\cite{597748ff-2020-3c20-9c06-e358280e09a2}) The old perspective of behaviorism explains behavioral patterns entirely with an external reward system; in contrast, Noam Chomsky's cognitive psychology attributes the driving force of human intelligence to internal representations like desires, beliefs, and purpose. To this date, this view from cognitive psychology is still agreed upon by scholars, while behaviorism theories have vanished. (\cite{reboot})

The paradigm of foundation models, in a sense, is like dictating a child's behavior by raising them in a controlled environment. Every model starts from a bunch of random numbers and then learns everything from data by maximizing rewards or minimizing loss. Although so far we've seen this approach make big progress, it is an extremely resource-intensive process, and we must remember that this is not how the human mind works. We should look into ways to design models that are capable of internalizing values and generating internal representations of these values. Structuring experiences and learnings to create a unique internal logic is a crucial ``human'' aspect of human intelligence. Artificial intelligence should also be able to replicate its user's internal representation in the training/inferencing process. Most large language models might be able to adapt to a user's response, but they do so by appending the user's previous feedback to future prompts. Internally, it's just the same model processing longer inputs with more added context. Compared to the human mind's capability of generating internal representations, this is extremely inefficient and again adds to the already enormous resources required to run these models.
\subsection{Trust}
{\itshape TODO: Talk about why people lose their trust on AI? How can we rebuild trust? Can we reimagine our relationship with AI? Perhaps more private, transparent and tactile experience.}

\subsection{Participatory AI}
{\itshape TODO: A more democratized process of AI development. Data-centric AI (\cite{Liang2022,10.1145/3411764.3445518})}

\section{Observations}
\subsection{AI is not good enough}
By saying that AI would benefit the world because it can solve cancer, climate change and poverty, popular AI advocative sentiment usually implies that if we solve AI, then we essentially solve everything. However, the actual implication of a solved world is far beyond general prosperity and panacae. Philosopher Nick Bostrom describes the condition of a solved world as \textit{Technology Maturity}, meaning that with AI we would have all the technologies we can possibly have and use them to solve whatever problem that remains. (\cite{bostromdeeputopia}) When we look into what exactly these technology are, you will see that we are still so far away from technology maturity.

{\itshape TODO: Examples. Brain-computer interface, solving AI is almost contingent on solving the human brain, which we are still far. }
\subsubsection{Language alone is not enough}
\subsection{Is AI ever going to be good enough?}
{\itshape TODO: Can we have human-level intelligence machine without inducing moral-status? Sentience machine (\cite{bostromdeeputopia})}
\subsubsection{Do we really want AI to be that good?}
{\itshape TODO: Purpose? Return of Malthusian economy. Extreme inequality -- all earning goes to capital. Reference Sec. \ref{econ}}
\subsection{AI is machine, not human}
The discussion of the long-term utopian (or dystopian) vision and speculation of AI is beyond the scope of this design work. We are designing for the world with current technologies and socioeconomical conditions. However, extrapolating the extreme future informs us a lot about what the currrent vesion of AI we have actually is and how it sits in the imaginable roadmap of the technology. Only once we have those understandings can we design for the AI we have instead the AI we envision.

Following previous discussions, we know that the current version of AI we have is still far away from what's promised. That is, the technology itself is still not good enough. However, in the discipline of design, we are designing the derivative product or human interface of AI \textit{as if} the technology was that good. The perimeter of the very limited capability of models is intentionally invisible. The intrinsic downfalls and externalities are hidden. In pursuing ease-of-use and ``seamless user experience'', user-configurablity is completely abandoned. It seems that designers want users to believe that the models can figure out what exactly what's needed where in fact more means of user input could potentially improve the performance dramatically. The prevalent chatbot interface gives up on informing the user how models work, essentially depriving user's oppurtunity to make models work better for them.

Chatbot is a human-human interface that lives on computing devices, not a human-computer interface. One of the implication of technology maturity is that whatever intelligence machine we come up with, we can interact with it in a manner extremely similar to human-human interaction. AI, as it is now, is a sort of intelligence machine that is still far from technology maturity. However, that interface that we came up with is the one that is supposed to make sense only once technology maturity is reached. In designing a interface like chatbot, the desginer forgot the fact that AI is still fundamentally a machine. Machines need to be operated in a certain way to yield maximum performance and efficiency. Operating a machine requires that the user has some technical knowledge. A well-planned learning curve is what makes a human interface usable. Typing on a keyboard is a learning curve. Scrolling, pinch-to-zoom, long-press on a touch screen all demand leanring. For a technology like AI that is still novel to the public, opting for a human-like interface that has close to no learning curve seems more like evading the hard work rather than being considerate.

In every other technology humanity has ever seen before, its human interface is always one that bridges the needs of users and the limitations of the technology. The purpose of human interface is to make the technology works for the human, hence the interface has to work for both the technology and the human. Chatbot, however, is one that only works for human but takes no consideration for the intrinsic properties and operational parameters of the models.

A good human interface informs the user the capabilities (and limitations) of the technology, best operational practices and potential means of trouble-shooting. Every usable machine in the history has a human interface that accomplishes these things. When a machinist is operating a laith, they can observe whether the power, gear ratio and cutting tool are making a clean cut. If they want to change the speed of the rotor, it's clear that the most preferable way is to change the gear ratio. If the cut is not clean even under optimal rotational speed, they would know maybe the cutting tool needs to be sharpened. Using a laith requires a somewhat steep learning curve, but at least there is one. On a desktop computer interface, the relative size of the menu bar vaguely indicates the capacity for multitasking. You can easily drag and resize each window to have the best combination of applications running. If the computer becomes slow and laggy, most people can easily induce that they should close some windows to keep everything running smoothly. These are all examples of how a human interface bridges the gap between what the user wants to do and what the technology can do. 

What the chatbot interface assumes is that the user can do all of these by giving more instructions in natural language to correct or optimize the model's performance. If the AI technology we have is good enough in terms of its cognitive capabalities (see Section \ref{cognition}), this is a quite viable assumption. However, our AI is still not there and this chatbot interface only results in prompts augmented with more and longer prompts, which ultimately exceeds the context window of the model and makes it disoriented. Chatbot is an interface for the AI that can do everything (or, one that is very close to reach technology maturity), but it is an extremely ineffective and inefficient way to ``operate'' today's large language models.

{\itshape TODO: Why augmenting chatbot instead of radical new design. Because it's langague model. Because language is the primary way of communication}

\section{Design Strategy}
This work aims to challenge the current AI paradigm by creating tangible interactions embedded with data-informed performances. The user might have to observe a little on-device performance or interact with the object to complete certain exercise in order to access the product they wish to use. By creating surprising moments in the expected user experiences of AI products, users/spectators are prompted to have a deep reflection of the unseen cost/benefit of AI. These ``surprising moments'' shall be consisted of small, physical performances happening on the designed artifacts. The combination of tangible interaction and physical artifacts is a sharp contrast of people's common understanding of AI as purely ditigal softwares, this contrast is employed here as a tool to stimulate a different perspective.

There are two main goals of the on-device performances: to inform the spectator of the hidden cost and logistics of the current AI paradigm and to provide a peek into the alternatives. The performances are supposed to be tightly integrated with the interactive experiences to emphasize the complication of the entanglement of these issues. 

When I think about how this thesis can be manifested into artifacts, I see several seemingly contradictory or incompatible paths to go down. I do not intend to situate this work within any of these measures, but I think it's a worthwhile effort to discuss the potentials of this work that I've thought about.
\begin{description}
    \item[Issues/Alternatives.] The main objective of this work is to challenge the current AI paradigm; the major means to achieve that is by presenting the issues and the alternatives. However, focusing on issues or alternatives might lead to very different research and design. Focusing on issues means gathering and presenting information; focusing on alternatives, on the other hand, means more exploration of novelty.
    \par The danger of focusing on issues is that it paints a bleak world without sparking new ideas, which one can argue is meaningless. Focusing on alternatives lies on a fine-line of sacrificing criticality and simply becomes another design project that fuels the AI hype.
    \item[Critical/Functional.] The duality of issues/alternatives is inherently tied to making critical artifacts or functional products. To propose a picture of the preferable alternatives is to design for utilities that the audiences can understand. To be critical about the topics, however, is to think of ways that engage audiences in the criticism. At the end of the day, the designed objects will be functional in some ways, but how much of the functions provide utility and how much of them provide criticism is for me a nuanced balance.
    \item[Orchestration/Exaggeration.] Should the presentation of the issues and alternatives be orchestrated coherently in the designed artifacts or should a exaggerated fantasy be presented to encapsulate everything?
\end{description}
\subsection{Dissection}
\subsection{Analogy}
{\itshape Dotcom bubble, internet, even older technology, electricity?}
\subsection{Orchestration}
\subsection{Speculation}
\newpage
\printbibliography[heading=bibintoc]
\end{document}